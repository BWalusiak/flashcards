% Preamble
\documentclass[11pt]{article}

% Packages
\usepackage{amsmath}
\usepackage{float}
\usepackage[T1]{fontenc}
\usepackage{lmodern}
\usepackage[polish]{babel}
\usepackage[utf8]{inputenc}
\usepackage{polski}
\usepackage{hyperref}
\usepackage{graphicx}
\setlength{\parindent}{0pt}

% Document
\begin{document}

    \title{Projekt SPIN nr 2 - CorpoFiszki}
    \author{Bartosz Walusiak, Rafał Łyżwa}
    \date{\today}
    \maketitle

    \tableofcontents

    \newpage

    \section{Opis projektu}\label{sec:description}
    \subsection{Cel}\label{subsec:target}
    Cel projektu:\\
    Wykorzystanie środowiska processing do stworzenia prototypu aplikacji pomagającej w nauce imion współpracowników.

    \subsection{Założenia przyjęte w trakcie realizacji}\label{subsec:design-choices}
    Założenia:
    \begin{itemize}
        \item Obsługa aplikacji za pomocą myszki.
        \item Import danych użytkownika za pomocą pliku .csv.
    \end{itemize}

    \section{Instrukcja obsługi}\label{sec:user-manual}
    \subsection{Uruchomienie aplikacji}\label{subsec:setup}

\end{document}